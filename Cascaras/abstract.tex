%---------------------------------------------------------------------
%
%                      abstract.tex
%
%---------------------------------------------------------------------
%
% Contiene el cap�tulo del resumen.
%
% Se crea como un cap�tulo sin numeraci�n.
%
%---------------------------------------------------------------------

\chapter{Abstract}
\cabeceraEspecial{Abstract}

\begin{FraseCelebre}
\begin{Frase}

\end{Frase}
\begin{Fuente}

\end{Fuente}
\end{FraseCelebre}

The target of this Project is developing and designing a personal capoeira trainer with the motion capture technology of \texttt{Kinect}. The created application has been given the name of VIC (\textit{Virtual Instructor of Capoeira}).

In order to understand and address this work correctly, the existing systems and technologies for motion capture have been reviewed. The advantages of using the device Kinect have been presented and some previous research studies with positive results have been mentioned. Moreover, new contributions made by this project on the field of \textit{Reactive Virtual Trainers (RVT)} have been highlighted, since it has led to a diversity of solutions for dancing, martial arts and physical activity in general. 

This project presents a personal training system of capoeira, a Brazilian martial art, defined as a mixture of fighting sports and dancing. 
This environment is thought for the apprentice to learn different capoeira movements without the need of having a teacher in person. The training consists on imitating some movements done by experts and which were previously recorded. Moreover, the students?�level can be chosen, whether they are beginners or advanced trainees. In this way, the apprentice learns progressively how to make the movements.

This low?priced and intuitive system tracks each learner?s movement. Likewise, the system will be easily adapted for the recovery of an injured part of the body or the analysis of athletes in sports medicine. After the study undertaken, the application functioning was considered fast and offers very accurate corrections of the movements made wrongly. For all these reasons, we can affirm that this environment has an excellent present and a hardly promising future.


\paragraph{Key Words:}Kinect, capoeira, motion capture, Mocap, Reactive Virtual Trainer, RVT, Unity 3D, personal trainer, comparison and correction of movements.


%
\endinput
% Variable local para emacs, para  que encuentre el fichero maestro de
% compilaci�n y funcionen mejor algunas teclas r�pidas de AucTeX
%%%
%%% Local Variables:
%%% mode: latex
%%% TeX-master: "../Tesis.tex"
%%% End:
