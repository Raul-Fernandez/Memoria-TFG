%---------------------------------------------------------------------
%
%                      resumen.tex
%
%---------------------------------------------------------------------
%
% Contiene el cap�tulo del resumen.
%
% Se crea como un cap�tulo sin numeraci�n.
%
%---------------------------------------------------------------------

\chapter{Resumen}
\cabeceraEspecial{Resumen}

\begin{FraseCelebre}
\begin{Frase}

\end{Frase}
\begin{Fuente}

\end{Fuente}
\end{FraseCelebre}


La investigaci�n en \textit{Reactive Virtual Trainers (RVT)} basados en el uso de \texttt{MS Kinect} ha dado lugar a una diversidad de soluciones para la danza, las artes marciales y el ejercicio f�sico en general. Este art�culo presenta un prototipo de sistema para el entrenamiento de capoeira, un arte marcial brasile�o a caballo entre los deportes de lucha y la danza.
Este entorno est� pensado para ser utilizado por el alumno para que pueda aprender a realizar diferentes movimientos de capoeira sin la necesidad de que est� presente un profesor. El entrenamiento consiste en imitar una serie de movimientos, los cuales han sido previamente capturados por varios expertos en el arte marcial. El entrenamiento virtual se desarrolla acorde al nivel que posea el alumno, ya sea principiante, aprendiz o avanzado. De este modo, el alumno va aprendiendo a realizar los movimientos de forma progresiva.


\endinput
% Variable local para emacs, para  que encuentre el fichero maestro de
% compilaci�n y funcionen mejor algunas teclas r�pidas de AucTeX
%%%
%%% Local Variables:
%%% mode: latex
%%% TeX-master: "../Tesis.tex"
%%% End:
