%---------------------------------------------------------------------
%
%                      resumen.tex
%
%---------------------------------------------------------------------
%
% Contiene el cap�tulo del resumen.
%
% Se crea como un cap�tulo sin numeraci�n.
%
%---------------------------------------------------------------------

\chapter{Resumen}
\cabeceraEspecial{Resumen}

\begin{FraseCelebre}
\begin{Frase}

\end{Frase}
\begin{Fuente}

\end{Fuente}
\end{FraseCelebre}

Este proyecto tiene como meta el desarrollo y dise�o de un entrenador personal de capoeira con la tecnolog�a mocap (Motion Capture) de Kinect.

Para entender y enfrentarse de una manera correcta a este trabajo, se han revisado los diferentes sistemas y tecnolog�as que existen para la captura de movimientos. Se han expuesto las ventajas que han llevado a utilizar el dispositivo Kinect y se han mencionado varios estudios de investigaci�n previos, con resultados favorables. Por �ltimo, se ha hecho hincapi� en las novedades que aporta este proyecto en el �mbito de los \textit{Reactive Virtual Trainers (RVT)}, ya que hace sentir al usuario que se encuentra en un entorno m�s real y dedicado.     

La investigaci�n en \textit{Reactive Virtual Trainers} basados en el uso de \texttt{ Kinect} ha dado lugar a una diversidad de soluciones para la danza, las artes marciales y el ejercicio f�sico en general. Este proyecto presenta un sistema de entrenamiento personal de capoeira, un arte marcial brasile�o, definido como una mezcla entre los deportes de lucha y la danza.
Este entorno est� pensado para ser utilizado por el alumno para que pueda aprender a realizar diferentes movimientos de capoeira sin la necesidad de que est� presente un profesor. El entrenamiento consiste en imitar una serie de movimientos, los cuales han sido previamente capturados por varios expertos en el arte marcial. El entrenamiento virtual se desarrolla acorde al nivel que posea el alumno, ya sea principiante, aprendiz o avanzado. De este modo, el alumno va aprendiendo a realizar los movimientos de forma progresiva.

Como conclusi�n, cabe mencionar que el entorno desarrollado se adapta a las posibilidades de cada alumno, permitiendo una correcta evoluci�n. Se trata de un sistema de bajo coste, intuitivo y que lleva a cabo un control del alumno, lo que posibilita como trabajo futuro, la gesti�n de nuevos usuarios y, as�, tener un registro sobre el avance conseguido de cada movimiento. Del mismo modo, el sistema se podr� adaptar f�cilmente para la rehabilitaci�n de alg�n miembro da�ado o el an�lisis de los atletas en la medicina deportiva. Tras el estudio realizado con algunos usuarios, muestran que el funcionamiento de la aplicaci�n es r�pida, adem�s al realizar el entrenamiento, las correcciones de los movimientos ejecutados de forma errona se acercan bastante a la realidad. Con todo esto, se concluye que este entorno tiene un excelente presente y un futuro muy prometedor.
\paragraph{Palabras clave:}Kinect, capoeira, captura movmiento, Mocap, Reactive Virtual Trainer, RVT, Unity 3D, entrenador personal.

%
\endinput
% Variable local para emacs, para  que encuentre el fichero maestro de
% compilaci�n y funcionen mejor algunas teclas r�pidas de AucTeX
%%%
%%% Local Variables:
%%% mode: latex
%%% TeX-master: "../Tesis.tex"
%%% End:
