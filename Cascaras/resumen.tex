%---------------------------------------------------------------------
%
%                      resumen.tex
%
%---------------------------------------------------------------------
%
% Contiene el cap�tulo del resumen.
%
% Se crea como un cap�tulo sin numeraci�n.
%
%---------------------------------------------------------------------
\clubpenalty=10000
%\widowpenalty=10000
\chapter{Resumen}
\cabeceraEspecial{Resumen}

\begin{FraseCelebre}
\begin{Frase}

\end{Frase}
\begin{Fuente}

\end{Fuente}
\end{FraseCelebre}

Este proyecto tiene como meta el dise�o y desarrollo de un entrenador personal de capoeira con la tecnolog�a de captura de movimiento de \texttt{Kinect}. La aplicaci�n utilizar� el nombre de VIC (\textit{Virtual Instructor of Capoeira}).

Para entender y abordar de una manera correcta este trabajo, se han revisado los diferentes sistemas y tecnolog�as que existen para la captura de movimientos. Se han expuesto las ventajas que han llevado a utilizar el dispositivo \texttt{Kinect} y se han mencionado varios estudios de investigaci�n previos con resultados favorables. Adem�s, se ha hecho hincapi� en las novedades que aporta este proyecto en el �mbito de los \textit{Reactive Virtual Trainers (RVT)}, ya que ha dado lugar a una diversidad de soluciones para la danza, las artes marciales y el ejercicio f�sico en general. 


Este proyecto presenta un sistema de entrenamiento personal de capoeira, un arte marcial brasile�o, definido como una mezcla entre los deportes de lucha y la danza.
Este entorno est� pensado para que el alumno pueda aprender a realizar diferentes movimientos de capoeira sin la necesidad de que est� presente un profesor. El entrenamiento consiste en imitar una serie de movimientos realizados por expertos y que han sido previamente grabados. Adem�s, se puede escoger el nivel del alumno, ya sea principiante, avanzado o experto. De este modo, el alumno va aprendiendo de forma progresiva.


Este sistema de bajo coste e intuitivo lleva a cabo un seguimiento del alumno en cada movimiento. As�, el sistema se podr� adaptar f�cilmente para, por ejemplo, la rehabilitaci�n de alguna parte del cuerpo da�ada o en la medicina deportiva. Tras el estudio realizado, se observ� que el funcionamiento de la aplicaci�n es r�pido y ofrece unas correcciones muy precisas de los movimientos ejecutados de forma err�nea. Por todo ello, se puede afirmar que este entorno tiene un excelente presente y un futuro muy prometedor.


\paragraph{Palabras clave:}Kinect, capoeira, captura movimiento, Mocap, Reactive Virtual Trainer, RVT, Unity 3D, entrenador personal, comparaci�n y correcci�n de movimientos.

%
\endinput
% Variable local para emacs, para  que encuentre el fichero maestro de
% compilaci�n y funcionen mejor algunas teclas r�pidas de AucTeX
%%%
%%% Local Variables:
%%% mode: latex
%%% TeX-master: "../Tesis.tex"
%%% End:
