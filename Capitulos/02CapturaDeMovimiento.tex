%---------------------------------------------------------------------
%
%                          Cap�tulo 2
%
%---------------------------------------------------------------------

\chapter{Estado del arte}


%-------------------------------------------------------------------
\section{Historia captura de movimiento}
%-------------------------------------------------------------------
\label{cap2:sec:captura}

La captura de movimiento o motion capture, MOCAP, es el proceso de grabaci�n del movimiento de actores o animales para transferirlo al personaje digital. La tecnolog�a de captura de movimientos surgi� en biomec�nica, para el estudio de la marcha humana, pero pronto su aplicaci�n se extendi� a campos tan dispares como los videojuegos o la neurociencia.
%-------------------------------------------------------------------
\section{Tecnolog�a de captura de movimiento}
%-------------------------------------------------------------------
\TocNotasBibliograficas

Captura de movimientos �ptica \\
Captura de movimientos en v�deo o Markerless , LUZ ESTRUCTURADA de kinect\\
Captura de movimientos inercial
\medskip

%Y tambi�n ponemos el acr�nimo \ac{CVS} para que no cruja.

%Ten en cuenta que si no quieres acr�nimos (o no quieres que te falle la compilaci�n en ``release'' mientras no tengas ninguno) basta con que no definas la constante \verb+\acronimosEnRelease+ (en \texttt{config.tex}).




% Variable local para emacs, para  que encuentre el fichero maestro de
% compilaci�n y funcionen mejor algunas teclas r�pidas de AucTeX
%%%
%%% Local Variables:
%%% mode: latex
%%% TeX-master: "../Tesis.tex"
%%% End:
