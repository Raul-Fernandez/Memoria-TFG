%---------------------------------------------------------------------
%
%                          Cap�tulo 2
%
%---------------------------------------------------------------------



\chapter{Estado del arte}


%-------------------------------------------------------------------
\section{Historia captura de movimiento}





%-------------------------------------------------------------------



%-------------------------------------------------------------------
\label{cap2:sec:captura}

La captura de movimiento (del ingl�s motion capture, o tambi�n abreviada mocap) es una t�cnica de grabaci�n de movimiento, en general de actores y de animales vivos, y el traslado de dicho movimiento a un modelo digital, realizado en im�genes de computadora. Se basa en las t�cnicas de fotogrametr�a y se utiliza principalmente en la industria del cine de fantas�a o de ciencia ficci�n, en la industria de los videojuegos o tambi�n en los deportes, con fines m�dicos. En el contexto de la producci�n de una pel�cula, se refiere a la t�cnica de almacenar las acciones de actores humanos, y usar esa informaci�n para animar modelos digitales de personajes en animaci�n 3D.
Actualmente la captura de movimiento es el m�todo m�s usado ya sea en el cine o en la industria de los videojuegos para llegar lo m�s posible a la realidad misma.

Principalmente la captura de movimiento se instal� en los videojuegos para mejorar el realismo y la dinamica de los mismos. Para realizar las capturas muchas veces las compa�ias se asocian con marcas de modelado 3D como Naturalmotion, Autodesk entre otros \citet{wiki1}.






%-------------------------------------------------------------------


\section{Tecnolog�a captura de movimiento}


%-------------------------------------------------------------------
\TocNotasBibliograficas

Captura de movimientos �ptica 


Captura de movimientos en v�deo o Markerless , LUZ ESTRUCTURADA de kinect


Captura de movimientos inercial
\medskip

%Y tambi�n ponemos el acr�nimo \ac{CVS} para que no cruja.

%Ten en cuenta que si no quieres acr�nimos (o no quieres que te falle la compilaci�n en ``release'' mientras no tengas ninguno) basta con que no definas la constante \verb+\acronimosEnRelease+ (en \texttt{config.tex}).




% Variable local para emacs, para  que encuentre el fichero maestro de
% compilaci�n y funcionen mejor algunas teclas r�pidas de AucTeX
%%%
%%% Local Variables:
%%% mode: latex
%%% TeX-master: "../Tesis.tex"
%%% End:
