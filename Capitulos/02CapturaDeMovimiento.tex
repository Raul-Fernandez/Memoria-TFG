%---------------------------------------------------------------------
%
%                          Cap�tulo 2
%
%---------------------------------------------------------------------



\chapter{Estado del arte}


%-------------------------------------------------------------------
\section{Historia captura de movimiento}





%-------------------------------------------------------------------



%-------------------------------------------------------------------
\label{cap2:sec:captura}
La captura de movimiento (abreviada Mocap, en ingl�s Motion Capture) es el proceso por el cual el movimiento, ya sea de objetos, animales o mayormente personas, se traslada a un modelo digital 3D.

En la actualidad, esta t�cnica llamada fotogrametr�a, se utiliza en la industria del cine y de los videojuegos, ya que facilita mucho la labor de los animadores al realizar un perfecto modelado d

La captura de movimiento (abreviada Mocap, en ingl�s Motion Capture) es una t�cnica de grabaci�n de movimiento, en general de actores y de animales vivos, y el traslado de dicho movimiento a un modelo digital, realizado en im�genes de computadora. Se basa en las t�cnicas de fotogrametr�a y se utiliza principalmente en la industria del cine de fantas�a o de ciencia ficci�n, en la industria de los videojuegos o tambi�n en los deportes, con fines m�dicos. En el contexto de la producci�n de una pel�cula, se refiere a la t�cnica de almacenar las acciones de actores humanos, y usar esa informaci�n para animar modelos digitales de personajes en animaci�n 3D.
Actualmente la captura de movimiento es el m�todo m�s usado ya sea en el cine o en la industria de los videojuegos para llegar lo m�s posible a la realidad misma.

Principalmente la captura de movimiento se instal� en los videojuegos para mejorar el realismo y la dinamica de los mismos. Para realizar las capturas muchas veces las compa�ias se asocian con marcas de modelado 3D como Naturalmotion, Autodesk entre otros \citet{wiki1}.



La captura de movimiento (Motion Capture, en ingl�s) es el proceso de digitalizar
movimientos reales de objetos o personas y a�adir, dichos movimientos, a modelos 3D.
Actualmente, en el cine se usa mucho esta t�cnica, ya que facilita el trabajo de los
animadores. Otro campo en el que esta t�cnica es puntera, sin duda, es en el de los
videojuegos. Para ofrecer una experiencia m�s realista a los usuarios, se utilizan t�cnicas
de captura de movimiento para naturalizar los movimientos de los personajes de los
juegos.
La t�cnica de animaci�n de personajes empleando la captura de movimiento, tal y
como se conoce ahora, tiene sus inicios en la d�cada de los 70 donde ya se comenzaba a
utilizar en pel�culas de animaci�n. Pero es en la d�cada de los 30 cuando se empez� a
indagar sobre la captura de movimiento.


%-------------------------------------------------------------------


\section{Tecnolog�a captura de movimiento}


%-------------------------------------------------------------------
\TocNotasBibliograficas

Captura de movimientos �ptica 


Captura de movimientos en v�deo o Markerless , LUZ ESTRUCTURADA de kinect


Captura de movimientos inercial
\medskip

%Y tambi�n ponemos el acr�nimo \ac{CVS} para que no cruja.

%Ten en cuenta que si no quieres acr�nimos (o no quieres que te falle la compilaci�n en ``release'' mientras no tengas ninguno) basta con que no definas la constante \verb+\acronimosEnRelease+ (en \texttt{config.tex}).




% Variable local para emacs, para  que encuentre el fichero maestro de
% compilaci�n y funcionen mejor algunas teclas r�pidas de AucTeX
%%%
%%% Local Variables:
%%% mode: latex
%%% TeX-master: "../Tesis.tex"
%%% End:
