%---------------------------------------------------------------------
%
%                          Cap�tulo 1
%
%---------------------------------------------------------------------
\addtocounter{chapter}{-1} 
\chapter{Introduction}


\label{cap1:sec:introduction}


Nowadays, motion capture systems have become an essential tool in the film and video game industry, since they make it easier for the animators and developers to create characters. With this technology, more realistic and precise movements are generated in less time, being the total cost lower than that of the old systems. These systems can be particularly useful, not only in the two previously?mentioned fields, but also in medical diagnosis, study of prototypes of machines, analysis of athletes in sports medicine, recovery of motor skills by disable people, etc. All this can be done with a household mocap tool, at a low price, without the need of using special suits, as in the case of \texttt{Kinect}.

The Reactive Virtual Trainers (RVT) is a technology that has been used for some years for dancing, martial arts and physical activity in general. Considering this functionality, we decided to bring together the practice of the African?Brazilian martial art, known as capoeira� and the \texttt{Kinect} motion capture system, in this way developing a personal trainer for this sport. All this has been possible due to the willingness of professionals at the Abad�?Capoeira school, who let us record them for the present project.

One of the main characteristics of capoeira, as in many other martial arts, is that a given movement needs to be done repeatedly in order to improve progressively both its trajectory or speed and strength. However, particularly in the case of new learners, it is difficult to observe if they improve without the constant supervision of a trainer, which it is almost always impossible. Therefore, the developed application of a personal trainer would make up for this lack.

With the purpose of developing a personal trainer accurately and thoroughly, firstly, we made some visits to the Asociaci�n Cultural Deportiva Rio (Sports Cultural Association Rio), where capoeira experts gather, to determine which movements would be optimally captured by \texttt{Kinect}. Later, the chosen movements were recorded several times so as to include the different capoeira techniques to the virtual trainer.

Research on RVT shows that users working with a screen focus more on moving objects, music, colours, movements, etc. than they do when working with a trainer in a gym. Therefore, greater progression has been observed in training done with RVT technology, since there is a higher competitiveness to make movements with a more precise technique. Another important advantage of \texttt{Kinect} with a personal trainer is that this technology can be used anywhere, such as at home, a residence or a hospital to do rehabilitation, among others. Consequently, it is accessible to every type of public, without the need of moving to the place where the activity usually takes place. 

In order to reproduce a more realistic and dynamic environment, different avatars were developed with a fairly human appearance. Furthermore, two different settings were created, each of them adapted according to their purpose. The first one consists of a training atmosphere, as a gym. The second one focuses on the teacher and the animators, representing a Brazilian beach. In this way, we managed to create an application which allows the user to feel that they are in a more realistic environment.



\section{Work Objectives}

RVT systems developed through mocap mainly focus on the realistic representation of how to help users to train and perfect the techniques. With this purpose, they are based on the comparison between the movements made by themselves and those previously recorded by an expert. Following these guidelines, some aims were set to be achieved in this Final Project (FP):


\begin{itemize}
	\item To identify the limits of the device \texttt{Kinect} to correctly achieve a motion capture of capoeira.
	
	
	\item To implement and design an application, with the help of a game engine, that simulates a virtual trainer to correct the users? movements.
	
	\item To do an analysis of capoeira movements captured with \texttt{Kinect} and, thus, classify those which were best registered for the training.
	
	
\end{itemize} 

\section{Project/Work Plan}


This project has been made in three phases: specification, implementation and documentation. 

The first stage was devoted to establish the objectives and the extent of the FP, being necessary to organise the dates to meet the tutors and so determine the following of the project.

Later, in the second stage, the specifications set in the previous phase are put into practice. This stage is then divided into three steps: the first one was choosing the environment to be used so as it adjusted to \texttt{Kinect} technology and to our way of working; the second step consisted on creating the application with its 3D environment, adding the comparison logic of movement and incorporating different avatars; and, in the last step, one the application was finished, the recordings of experts in capoeira were made, analysing and adjusting their movements to add them to the virtual trainer.

Finally, in the documentation phase, all the necessary information and content was gathered to develop the Final Project. This phase took place simultaneously with the implementation.

Below is a list of the tasks of each phase and their estimated dates.



\begin{itemize}
	\item Specification of the project
	\begin{itemize}
		\item Project Definition: 2015 - July 2016.
		\item Goal Setting: December 2016 - January 2016.
		\item Weekly meetings with tutors: December 2016 - June 2017.
	\end{itemize}
	\item Implementation of the project.
	\begin{itemize}
		\item Choice of development environment: December 2016.
		\item Creation of the virtual trainer: February 2017 - May 2017.
		\item Avatars design: april 2017.
		\item Creating 3D Scenarios: May 2017.
		\item Recording with expert people: May 2017.
	\end{itemize}
	\item Project documentation.
	\begin{itemize}
		\item Documentation for the project: April 2017 - June 2017.
		\item Realization of memory: May 2017 - June 2017.
	\end{itemize}
\end{itemize}

%\begin{table}[htb]
%	\centering
%	\begin{tabular}{|c|l|l|}
%		\hline
%		Fases & Tareas del proyecto & Fecha estimada \\
%		\hline \hline
%		\multirow{4}{3cm}{Especificaci�n del proyecto} &
%		Definici�n del proyecto & 2015 - julio 2016    \\ \cline{2-3} 
%		& Plateamiento de los objetivos & 13/12/2016 - 15/10/2016  \\ \cline{2-3}
%		& Reuniones semanales con los tutores & 13/12/2016 - 15/06/2017\\ \cline{1-3}
%		\multirow{3}{3cm}{Implementaci�n del proyecto} &
%		Elecci�n del entorno de desarrollo& 13/12/2016  - 20/12/2016\\ \cline{2-3}
%		&Creaci�n del entrenador virtual & 22/02/2017  - 05/05/2017\\ \cline{2-3}
%		&Dise�o de avatares & 11/04/2017 - 14/05/2016\\ \cline{2-3}
%		&Creaci�n de escenarios 3D & 15/05/2016 - 22/05/2016\\ \cline{2-3}
%		&Grabaci�n con gente experta & 06/05/2017 - 13/05/2017 \\ \cline{1-3}
%		\multirow{2}{3cm}{Documentaci�n del proyecto}  &    Documentaci�n para el proyecto & 09/04/2017 - 15/06/2017  \\ \cline{2-3}
%		& Realizaci�n de la memoria & 01/05/2017 - 15/06/2017 \\ \cline{1-3}
%	\end{tabular}
%	\caption{Tabla de plan de proyecto.}
%	\label{tabla:Plan_proyecto}
%\end{table}


\section{Memory Structure}

This project is structured in six chapters defined as follows:


\begin{itemize}
	\item In chapter one, described the introduction to the project, the main objectives and the work plan.
	
	
	\item In chapter two, virtual trainers and the history and different methods of motion capture are described.
	
	\item Chapter three encompasses the technologies used, both hardware and software, and the decision taken for the chosen environment is justified.
	
	\item In chapter four, the entire development of the project is described, explaining the previous study in distinct sections and the necessary implementations for the creation of a virtual capoeira trainer. Moreover, the transition of scenes in the application is explained.
	
	\item In chapter five, the conclusion drawn from the project is presented and suggestions for the future are offered.
	
	\item In chapter six, the distribution of tasks between the two members of the group is described.
	
\end{itemize} 


% Variable local para emacs, para  que encuentre el fichero maestro de
% compilaci�n y funcionen mejor algunas teclas r�pidas de AucTeX
%%%
%%% Local Variables:
%%% mode: latex
%%% TeX-master: "../Tesis.tex"
%%% End:
