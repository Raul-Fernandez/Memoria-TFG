%---------------------------------------------------------------------
%
%                          Cap�tulo 7
%
%---------------------------------------------------------------------

\chapter{Distribuci�n de trabajo}

En el presente TFG la distribuci�n de trabajo por parte de los miembros del equipo de desarrollo ha sido bastante ecu�nime.

  
En una primera instancia fue necesario realizar una investigaci�n conjunta sobre el campo de trabajo. El cual comprende el uso de la inform�tica y nuevas tecnolog�as aplicadas a la captura de movimiento, que al ser un tema totalmente novedoso para los desarrolladores, implic� un estudio del estado del arte actual.  


Debido a las caracter�sticas del trabajo y los dispositivos necesarios para su desarrollo, al contar con recursos limitados, es decir, con �nicamente una \texttt{Kinect} y un equipo compatible con las caracter�sticas de la misma, resultaba complicado el desarrollo en paralelo. Por ello se adopt� la pol�tica de establecer reuniones diarias, en las que se trabajaba de manera intensiva en tareas conjuntas que permitieran la integraci�n en el sistema de los dispositivos mencionados. Estas reuniones  tuvieron lugar en la Facultad de Inform�tica de la Universidad Complutense de Madrid, espec�ficamente en la biblioteca de este centro.

El objetivo general del proyecto era el desarrollo de un entrenador personal virtual basado en el arte marcial afro-brasile�o capoeira. Para ello, la intenci�n consist�a en mostrar, mediante un escenario 3D, el entrenamiento realizado por un usuario para el aprendizaje de capoeira, siendo necesario exponer los movimientos del entrenador virtual con el fin de ser imitando. Inicialmente, se desconoc�a como se iba a llevar a cabo este objetivo global y abstracto, ya que requer�a la comprensi�n profunda del dispositivo \texttt{Kinect} y sus posibilidades. Uno de los aspectos m�s importantes del proyecto ha sido todo lo relacionado con la captura de movimientos. Debido a los problemas de conectividad de \texttt{Kinect} con los ordenadores y que al final solo funcionaba en un equipo, se decidi� de manera conjunta, estudiar y extraer los datos transmitidos por el sensor de \texttt{kinect}.

Tras un breve periodo inicial de aprendizaje en Unity ambos desarrolladores contaban con los conocimientos necesarios para abordar el escenario 3D. Una vez entendido el funcionamiento de la simulaci�n del \textit{cubeman}, de forma conjunta, se implement� la forma de reproducir los movimiento pregrabados, realizados por los desarrolladores, a la vez que simulaba el movimiento en tiempo real de un usuario.

A partir de esta etapa del desarrollo se pudo realizar cierto reparto de tareas, pero siempre atado a las limitaciones de tener una �nica \texttt{Kinect}. Por una parte se destin� al dise�o de avatares con el programa \texttt{Fuse Character Creator de Mixamo} , a la vez que se elaboraba el dise�o de interfaz inicial de la aplicaci�n, y por otro lado, se enfoco en las animaciones de los avatares y la implementaci�n del an�lisis una vez realizada la comparaci�n del movimiento. Aunque se repartieron las tareas era necesario seguir la pol�tica de reuniones constantes para compactar las partes y comprobar su correcto funcionamiento.

Posteriormente, despu�s de conexionar las tareas anteriores, cada desarrollador dise�a un escenario para colocar a los diferentes avatares que aparecen en el proyecto, y de esta manera, poder diferencia la caracter�stica que les identifica.

Por ultimo, con la actuaci�n de ambos componentes del grupo, se hicieron grabaciones en la la escuela Abad�-Capoeira con gente experta en capoeira, para despu�s ajustar y catalogar los movimientos que se incorporaran al proyecto.


