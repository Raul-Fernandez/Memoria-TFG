%---------------------------------------------------------------------
%
%                          Conclusion
%
%---------------------------------------------------------------------


\addtocounter{chapter}{-1} 
\chapter{Discussion, Conclusions and Future Work }
\label{cap7:sec:Conclusions}


%-------------------------------------------------------------------
\section{Discussion}

As mentioned in the introduction, one of the aims of developing this system was that it could work with household technology, which severely limits the range of possibilities to be considered. The chosen settings for the system use only one camera provided by \texttt{Kinect} to capture the user's movement. This restricts the system capacity of noticing the most complex movements, such as those in which there is torso rotation, which may generate confusion among different members of the avatar who have axial symmetry (after the turn, the systems confuses one arm with the other and thinks that the user is still in front of it when they are actually with their back towards it). It happens similarly when analysing the depth of some movements. Hopefully, this project might be broaden in the future by using a greater number of cameras.

On the other hand, in some of the studied projects (e.g. \cite{Keerthy:Thesis:2012,Kyan:2015:ABD:2753829.2735951}) the users'�movements are analysed with more sofisticated techniques than those used in the present project. Although those techniques can be useful at higher levels of learning, when dealing with beginners in capoeira, there are not big differences with more simple techniques. This is because an apprentice's movements are more basic and, consequently, the comparison with the trainer is more basic too. Therefore, a more simple method of comparison of movements means two clear advantages: firstly, calculating it is less complex and quicker; and, secondly, the result of the comparison is easier to be translated into text in order to provide the apprentice with useful explanations on how to improve the techniques.


\section{Conclusions}

The current project has described the development of a virtual capoeira trainer from the previous recording of an expert's movement done with the device \texttt{Kinect}. On the other hand, the user, with the same mentioned device, will make a motion capture which will be used to be compared with that of the virtual trainer. Once both versions are compared, an adequate explanation is offered so as to correctly make those movements that were made inaccurately. Throughout this process, it must be emphasized the capacity of the device \texttt{Kinect} to capture the movements correctly, a relevant fact if, besides, we consider its low market price.

Possible benefits of using a virtual trainer must be highlighted, ranging from the comfort of working at home to obtaining a detailed, personal and immediate feedback. Likewise, it is a training system based on playing, which fosters learning and motivation, as can be observed in the student's scene, where the movement must be made correctly. It allows setting the difficulty of the movement comparison, which makes it possible to adapt it to the different students$'$ needs and capacities. It is an intuitive environment, both for the teacher and the student, which can be complemented with the help section offered by the application. Moreover, its household use at a low price makes it accessible to a broader public.

A detailed evaluation with learners and capoeira experts has not been made yet. Nonetheless, preliminary tests have actually been done to check the correct functioning of the application as well as the reliability of the comments made in order to correct its mistakes. These tests have been developed with a small group of users of different levels and the initial results show that, considering the level at which they have been done and with not too complex whole body movements, the application functioning is quick and the results of the mistakes analysis and explanations are fairly close to reality.



\section{Future Work}


This project has potential in the field of sports training, since a visual feedback is offered when comparing the user's movement with that of the expert. In this way, the user will be able to correct their technique if they make some mistake. 

A possible functionality to implement in the future would be the management of different users and the register of their progression. When the user logs in their account, the percentage of the improvement achieved in each movement will be shown.

Another future extension would be adding more functionality to the expert who is recording the movements. This feature will allow them to easily manage and modify the recorded movements.
Apart from this, for the movements analysis, we suggest the need of including more sophisticated methods similar to those used in \cite{Keerthy:Thesis:2012,Kyan:2015:ABD:2753829.2735951} to carry out a more accurate analysis of the movements that gives more useful results to advanced capoeira learners.

This procedure of movement analysis can be also applied to the medicine field to do different types of physical therapy. In that case, it would be useful to tell a patient with, for example, an arm injury if the movement they are doing in the rehabilitation is being correct or not.

Another factor to improve motion capture would be adding another \texttt{Kinect} so as to recognize movements that involve turning. A similar solution is proposed in \cite{gao2015leveraging} where two \texttt{Kinect} are used to solve this problem, although this option distances from its household use.

Finally, analysing the movements of more than one user simultaneously is also considered. In this way, it would be possible to reproduce similar situations to the pair work done in a capoeira \textit{roda}, where techniques of body to body fight are implemented.
