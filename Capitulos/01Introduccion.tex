%---------------------------------------------------------------------
%
%                          Cap�tulo 1
%
%---------------------------------------------------------------------

\chapter{Introducci�n}

\begin{FraseCelebre}
\begin{Frase}

\end{Frase}
\begin{Fuente}

\end{Fuente}
\end{FraseCelebre}

%\begin{resumen}

%\end{resumen}


%-------------------------------------------------------------------
\section{Introducci�n}
%-------------------------------------------------------------------
\label{cap1:sec:introduccion}



La capoeira es un arte marcial afro-brasile�o que combina facetas de danza, m�sica y acrobacias, as� como expresi�n corporal \cite{assuncao2005capoeira}. Fue desarrollado en Brasil por descendientes africanos con influencias ind�genas, probablemente a principios del siglo XVI. Es conocido por sus r�pidos y complejos movimientos, que utilizan distinas partes del cuepro (brazos, piernas, codos, rodillas, cabeza...) para ejecutar maniobras de gran agilidad en forma de patadas, fintas y derribos, entre otros. La capoeira como estilo de lucha incorpora movimientos bajos y barridos, mientras que en el �mbito deportivo se hace m�s �nfasis en las acrobacias y las demostraciones ritualizadas de habilidad. Se practica con m�sica tradicional de berimbau. 

Una de las caracter�sticas fundamentales de la capoeira, como de muchas otras artes marciales, consiste en la realizaci�n de manera repetida de un mismo movimiento para ir mejorando paulatinamente su ejecuci�n, tanto en su trayectoria como en su velocidad o su fuerza. Sin embargo, especialmente en el caso de los principiantes, resulta complejo apreciar si existe mejora sin la supervisi�n constante de un entrenador, lo cual no resulta posible en la mayor parte de las ocasiones. Por este motivo, se ha planteado el desarrollo de un entrenador de capoeira, inicialmente en un nivel b�sico, que, mediante la utilizaci�n de tecnolog�as que ya son de uso dom�stico (como son un ordenador personal y \texttt{Kinect}), supervise el proceso de entrenamiento de una persona que se inicie en el aprendizaje de la capoeira. 

\section{Objetivos}
Los objetivos a desempa�ar en este trabajo de fin de grado(TFG) son:
\begin{itemize}
	\item Realizar un estudio de las m�todos de captura de movimiento y trabajos relacionados con el fin de comprender la tecnolog�a de captura de movimiento. 
	\item Estudiar el funcionamiento y descubrir el alcance que puede tener \texttt{kinect}.
	\item Dise�ar una metodolog�a utilizando \texttt{kinect} para la comparaci�n de movimiento mediante un entrenador virtual aplicado a capoeira .
	\item Implementar y dise�ar una aplicaci�n en un entorno 3D que simule un entrenador virtual para que realice las correcciones de los movimiento de los usuarios.
\end{itemize} 

\section{Plan de Trabajo}
se puede dividir en fases como dise�o implementacion documentacion. a�adir tabla con fechas  decir que teniamos reuniones semanales para llevar un historico del progreso del poryecto etc.
\section{Estructura de la memoria}
Este trabajo se estructura en siete cap�tulos definidos de la siguiente manera:
\begin{itemize}
\item En el capitulo dos se describe los entrenadores virtuales, la historia de la captura de movimiento y los diferentes m�todos de captura de movimiento que existen.

\item En el capitulo tres se engloba las tecnolog�as utilizas, tanto hardware como software y se describe la toma de decisi�n de que entornos se han utilizado. 

\item En el capitulo cuatro describe todo el desarrollo del proyecto explicando en diferentes secciones el estudio previo y las implementaciones necesarias para la creaci�n del entrenador virtual de capoeira. 

\item En el capitulo cinco se explican todas las transiciones de escenas implementadas en la aplicaci�n.

\item En el capitulo seis se presenta la conclusiones obtenidas del proyecto y el trabajo futuro.

\item En el capitulo siete se describe la distribuci�n del trabajo por parte de los integrantes de grupo.
\end{itemize} 




% Variable local para emacs, para  que encuentre el fichero maestro de
% compilaci�n y funcionen mejor algunas teclas r�pidas de AucTeX
%%%
%%% Local Variables:
%%% mode: latex
%%% TeX-master: "../Tesis.tex"
%%% End:
