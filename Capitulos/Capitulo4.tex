%---------------------------------------------------------------------
%
%                          Cap?tulo 4
%
%---------------------------------------------------------------------

\chapter{Materiales y M?todos}


\label{cap4:sec:kinect}
En este capitulo se expondr?n los  materiales  y m?todos utilizados para el desarrollo del proyecto. En los siguientes apartados se explicar?n con detalle y de forma t?cnica las tecnolog?as y los dispositivos utilizados para realizar este proyecto.

%-------------------------------------------------------------------
\section{Software y hardware empleados}
%-------------------------------------------------------------------
\label{cap4:sec:Software y hardware empleados}
El software principal utilizado en este proyecto es Unity 5.5 (poner referencia), que es un motor de desarrollo de videojuegos.
Para la edici?n , compilaci?n y depuraci?n de la programaci?n de Unity se ha utilizado Visual Studio 2015 (poner referencia) con el lenguaje de programaci?n C\#.\\
\\
\comImpl{Aqui raul pones los programas de modelado de avatar. Animaciones de mixamo,.}\\
\\

El hardware principal usado en este proyecto es el dispositivo de captura de movimiento kinect v2 (referencia) y el cable de conexion(nose como se llama, refrencia).







\subsection{Entorno de desarrollo : Unity 3D}
%-------------------------------------------------------------------
\label{cap4:sec:Entorno de desarrollo : Unity 3D}
El objetivo de este proyecto es la captura y el an?lisis de movimiento relacionados con el arte marcial afro-brasile?o capoeira(?referencia?). Con esto en mente, se llega a la primera decisi?n de que plataforma escoger para el desarrollo de este proyecto.\\

Los entornos a elegir ser?an Visual Studio 2015, Unity 3D y Unreal Engine 4 (referencia). El primero descartado es desarrollar directamente en Visual Studio 2015 porque se busca crear tambi?n un escenario 3D y, tanto Unity como Unreal ,facilitan la creaci?n de estos escenarios.\\\\
En cuanto a la decisi?n de elegir entre Unity 3D o Unreal Engine 4 fue basada en la comunidad que hay detr?s de cada uno de ellos, y sobre todo, en lo que se quiere abarcar con este proyecto. Unreal suele ser utilizado por las empresas para juegos grandes y mas profesionales , mientras que Unity se puede aplicar a peque?as y grandes aplicaciones siendo su aprendizaje de este mas intuitivo que Unreal.
\\\\
Y por ultimo la elecci?n de lenguaje de programacion siendo el lenguaje de C\# una opci?n m?s sencilla para el desarrollo de este proyecto.(buscar diefrencias claras)


\section*{Comunidad de Unity}
 Video: 640x480 @30 fps
%-------------------------------------------------------------------
\subsection{Kinect V2}
%-------------------------------------------------------------------




\comImpl{Enlaces sobre las caracteristicas de Kinect}

\medskip
https://msdn.microsoft.com/library/jj131033.aspx\\
https://msdn.microsoft.com/library/dn782025.aspx\\
https://developer.microsoft.com/es-es/windows/kinect/hardware

%Y tambi?n ponemos el acr?nimo \ac{CVS} para que no cruja.

%Ten en cuenta que si no quieres acr?nimos (o no quieres que te falle la compilaci?n en ``release'' mientras no tengas ninguno) basta con que no definas la constante \verb+\acronimosEnRelease+ (en \texttt{config.tex}).


%-------------------------------------------------------------------
%\section*{\ProximoCapitulo}
%-------------------------------------------------------------------



% Variable local para emacs, para  que encuentre el fichero maestro de
% compilaci?n y funcionen mejor algunas teclas r?pidas de AucTeX
%%%
%%% Local Variables:
%%% mode: latex
%%% TeX-master: "../Tesis.tex"
%%% End:
